\documentclass[conference]{IEEEtran}
\IEEEoverridecommandlockouts
% The preceding line is only needed to identify funding in the first footnote. If that is unneeded, please comment it out.
\usepackage{cite}
\usepackage{amsmath,amssymb,amsfonts}
\usepackage{algorithmic}
\usepackage{graphicx}
\usepackage{textcomp}
\usepackage{xcolor}
\usepackage{hyperref}

\def\BibTeX{{\rm B\kern-.05em{\sc i\kern-.025em b}\kern-.08em
    T\kern-.1667em\lower.7ex\hbox{E}\kern-.125emX}}
\begin{document}

\title{
    Operating Systems for IoT Devices: \\
    Characteristics, Challenges, Attack Surfaces, \\ and OS Landscape
    \\
}

\author{
    \IEEEauthorblockN{Paul Christian Pienkny}
    \IEEEauthorblockA{
        \textit{Freie Universit\"at Berlin} \\
        IoT \& Security Seminar Report
    }
}

\maketitle

\begin{abstract}
    The report aims to provide a comprehensive overview of the current
    landscape of IoT operating systems, highlighting their differences and 
    unique features, as well as security challenges they face in the
    landscape of rapidly evolving IoT ecosystems.
    \\
\end{abstract}

\begin{IEEEkeywords}
    Operating Systems (OS), Internet of Things (IoT),
    Characteristics, Security Challenges, Attack Surfaces
\end{IEEEkeywords}

\section{Introduction}

    \subsection{Security Relevance of IoT Devices \& Operating Systems}

    \subsection{Objectives \& Research Questions}

    \subsection{Report Structure}

\section{Fundamentals}

    \subsection{IoT Devices \& Architectures}

    \subsection{Requirements for IoT Operating Systems}
        Resource constraints, real-time capabilities, \\
        (energy) efficiency\dots

    \subsection{Security Fundamentals}

        Threats, vulnerabilities, attack vectors \\
        in IoT environments\dots \\
        (general theoretical background)

\section{Overview of IoT Operating Systems}

    \subsection{Brief Introduction of Selected IoT Operating Systems}

        RIOT, FreeRTOS, (Zephyr, TinyOS,)\dots \\
        Buildroot(Linux), Yocto(Linux)\dots

    \subsection{Comparison of Selected Operating Systems}

        Use cases, (kernel) architecture, resource consumption, \\
        security features, real-time capabilities\dots

    \subsection{Known Vulnerabilities or Attacks on IoT Operating Systems}

        e.g., related to Privilege Escalation, Memory Protection\dots

\section{Security Features \& Requirements of IoT Operating Systems}

    \subsection{Security Principles}

        Avoidance Principle, Minimality Principle, \\
        Defense-in-Depth, Fail Secure, \\
        Secure-by-Design\dots \\
        (concrete guidelines and methods)

    \subsection{Integrity \& Authenticity}

        e.g., Secure Boot, Code Signing, (Firmware-)Updates\dots

    \subsection{Isolation \& Access Control}

        e.g., Memory Protection, Sandboxing, \\
        (Micro-)Kernel Architecture\dots

    \subsection{OS-Specific Security Mechanisms}

        e.g. RIOT Secure Boot, FreeRTOS/Zephyr Memory Protection\dots

\section{Challenges \& Research Directions}

    \subsection{Resource Constraints vs. Security Mechanisms}

    \subsection{Lifecycle Management \& Update Strategies}

    \subsection{Formal Verification, Microkernels \& Trusted Execution Environments}

    \subsection{Recent Trends in Research}

        e.g. OAT(Testing)\dots

        e.g., Edge Security, rBPF, WebAssembly, Trusted Execution Environments, \\
        Microkernel Isolation, Lightweight Cryptography\dots

\section{Conclusion \& Outlook}

    \subsection{Summary of Findings}

    \subsection{Future Developments \& Research Perspectives}

    \subsection{Remaining Open Questions \& Challenges}

%
% LITERATURE
%

\begin{thebibliography}{00}

\bibitem{oat-testing}
\href{https://doi.org/10.1109/SP40000.2020.00042}{OAT: Attesting Operation Integrity of Embedded Devices (2020, published)}

\bibitem{riot-iot-os}
\href{https://doi.org/10.1109/JIOT.2018.2815038}{RIOT: an Open Source Operating System for Low-end Embedded Devices in the IoT (2018, published)}
% Free Access: \href{https://ilab-pub.imp.fu-berlin.de/papers/bghkl-rosos-18-prepub.pdf}{Preprint Version}

\bibitem{trusted-embedded-linux}
\href{https://www.kernel.org/doc/ols/2007/ols2007v2-pages-79-86.pdf}{Trusted secure embedded Linux (2007, published)}

\bibitem{survey-layered-iot-2020}
\href{https://doi.org/10.1002/ett.3935}{A survey of security and privacy issues in the Internet of Things from the layered context (2020, published)}

\bibitem{survey-iot-os-2023}
\href{https://doi.org/10.48550/arXiv.2310.19825}{A Survey of the Security Challenges and Requirements for IoT Operating Systems (2023, unpublished)}

\bibitem{rtos-iot-os}
\href{https://doi.org/10.1109/WSCAR.2016.21}{Real Time Operating Systems for the Internet of Things, Vision, Architecture and Research Directions (2016, published)}

\bibitem{iot-firmware-security}
\href{https://doi.org/10.1007/s43926-023-00045-2}{A survey on IoT \& embedded device firmware security: architecture, extraction techniques, and vulnerability analysis frameworks (2023, published)}

\bibitem{iot-firmware-security}
\href{https://doi.org/10.1109/SOCA.2014.58}{IoT Security: Ongoing Challenges and Research Opportunities (2014, published)}
% Free Access: https://kalli.no/exre-site/pdf/dat159/iot/IoT+Security-+Ongoing+Challenges+and+Research+Opportunities.pdf

% Own references are added here by APA style, e.g.:
% \bibitem{pX} P. C. Pienkny, "Title of my own referenced work," Journal Name, vol. X, no. Y, pp. Z-ZZ, Year. + (Nothing for Books, "unpublished" or "in press" for published works)

% Efficient Isolation of Trusted Subsystems in Embedded Systems (Journal pp. 344-361)
% https://www.researchgate.net/profile/Mauro-Conti/publication/221273084_CED/links/55e3786108aecb1a7cc9cf29/CED.pdf#page=356

\end{thebibliography}

\vspace{12pt}

\end{document}
