\documentclass[conference]{IEEEtran}
\IEEEoverridecommandlockouts
% The preceding line is only needed to identify funding in the first footnote. If that is unneeded, please comment it out.
\usepackage{cite}
\usepackage{amsmath,amssymb,amsfonts}
\usepackage{algorithmic}
\usepackage{graphicx}
\usepackage{textcomp}
\usepackage{xcolor}
\usepackage{hyperref}

\def\BibTeX{{\rm B\kern-.05em{\sc i\kern-.025em b}\kern-.08em
    T\kern-.1667em\lower.7ex\hbox{E}\kern-.125emX}}
\begin{document}

\title{
    Operating Systems for IoT Devices: \\
    Characteristics, Challenges, Attack Surfaces, \\ and OS Landscape
    \\
}

\author{
    \IEEEauthorblockN{Paul Christian Pienkny}
    \IEEEauthorblockA{
        \textit{Freie Universit\"at Berlin} \\
        IoT \& Security Seminar Report - ALPHA
    }
}

\maketitle

\begin{abstract}
    The report aims to provide a comprehensive overview of the current
    landscape of IoT operating systems, highlighting their differences and 
    unique features, as well as security challenges they face in the
    landscape of rapidly evolving IoT ecosystems.  \\ \\
    (This document is only an ALPHA version of the upcoming report)
    \\
\end{abstract}

\begin{IEEEkeywords}
    Operating Systems (OS), Internet of Things (IoT),
    Characteristics, Security Challenges, Attack Surfaces, 
    Microcontroller Unit, Embedded Devices
\end{IEEEkeywords}

\section{Introduction}

    \subsection{Security Relevance of IoT Devices \& MCU-based Systems}

        "Brief context: Why especially MCU-based systems are a security concern\dots"

        Good Overview for Questions and Numbers.
        \cite{tanSoKWheresComprehensive}

        Reasons of Security Relevance for Embedded Systems.
        \cite{fournarisExploitingHardwareVulnerabilities2017}

        Reasons regarding intellectual property.
        \cite{bognarIntellectualPropertyExposure}

        Challenges with IoT Devices.
        \cite{azizalkabirSecuringIoTDevices2023}

        Security Requirements of IoT Systems (on MCUs).
        \cite{pearsonMisconceptionHardwareCost2019}

        Use Cases of RTOS in IoT.
        \cite{abdelsameaRealTimeOperating2016a}

        Severity of IoT System vulnerabilities.
        \cite{hossainHolisticAnalysisInternet2024}

    \subsection{Objectives \& Research Questions}

        Goal is to highlight differences, current problems and
        possible solutions regarding the security
        of (especially MCU-based) IoT operating systems on IoT devices. \\

        Separating between problems that are bound to Hardware/Architecture
        and problems that are bound to the Operating System.

    \subsection{Report Structure}

\section{Fundamentals \& Characteristics}

    "How do MCU-based IoT Systems work technically, what are their characteristics, \\
    and what are the general security fundamentals in this context?"

    \subsection{IoT Hardware Models \& Architectures}

        MCUs vs. MPUs - focus on MCUs, \\
        typical hardware architectures, \dots \\

        Differences in Architecture on ARM.
        \cite{ngabonzizaTrustZoneExplainedArchitectural2016a}
        
        Architecture Types and Features of OS for IoT.
        \cite{javedInternetThingsIoT2018}

    \subsection{Characteristics of MCU IoT Operating Systems}

        Resource constraints, real-time capabilities, \\
        (energy) efficiency\dots \\

        List of possible Hardware Limitations.
        \cite{tanSoKWheresComprehensive}

        Resource Exhaustion as a Security Issue.
        \cite{azizalkabirSecuringIoTDevices2023}

        What is an RTOS?
        Includes Architectural Layers of RTOS for IoT.
        \cite{abdelsameaRealTimeOperating2016a}

        Data for different specifications of IoT OSes.
        \cite{chandraOperatingSystemsInternet2016}

    \subsection{Security Fundamentals}

        Threats, vulnerabilities, attack vectors \\
        in IoT environments\dots \\
        (general theoretical background) \\

        Good example for Vulerabilities, Architectural Issues.
        \cite{tanSoKWheresComprehensive}

        Specifically Side-Channel-Attacks on MCUs.
        \cite{fournarisExploitingHardwareVulnerabilities2017}

        Common Weaknesses/Vulnerabilities:
        \cite{al-boghdadyPresenceTrendsCauses2021}

        Attacks on TEEs.
        Also Side-Channel-Attacks.
        Architectural Attacks.
        \cite{munozSurveyInsecurityTrusted2023}

        Common Threats to IoT Devices.
        Also Side-Channel-Attacks, Network Attacks.
        Security Issue Layers.
        \cite{azizalkabirSecuringIoTDevices2023}

        Kinds of Securities (Not Layers).
        \cite{pearsonMisconceptionHardwareCost2019}

        Constraints on employing conventional security solutions.
        Attack Surfaces \& Device, Communication, Service Vulnerabilities.
        \cite{hossainHolisticAnalysisInternet2024}

    \subsection{Brief Introduction of Selected MCU-based IoT Operating Systems}

        RIOT, FreeRTOS, Zephyr, Mbed OS, Contiki, TinyOS \\

        Windows, Linux, Android are not in the list,
        because they not MCU-based.
        Also good Overview of Operating Systems.
        \cite{antonyReviewIoTOperating2020}

        Another good Overview of Operating Systems.
        \cite{al-boghdadyPresenceTrendsCauses2021}

        General informations about various OSes.
        \cite{javedInternetThingsIoT2018}


    \subsection{Comparison of Selected Operating Systems}

        Use cases, (kernel) architecture, resource consumption, \\
        security features, real-time capabilities\dots \\

        Use cases and characteristics of multiple OSes.
        \cite{javedInternetThingsIoT2018}

        Scheduling on different OSes.
        Also Networking, Memory Management.
        \cite{javedInternetThingsIoT2018}

        Overview of OSes and from into RTOS.
        Figure 3 gives a good overview over the different OSes/Groups.
        Also includes complete comparison chart for (probably all)
        different OSes.
        \cite{chandraOperatingSystemsInternet2016}

\section{Security Requirements \& Principles for IoT Operating Systems}

    "What security requirements do these systems have to fulfill, \\
    and which principles guide their design?"

    \subsection{Security Principles for MCU-based Systems}

        Avoidance Principle, Minimality Principle, \\
        Defense-in-Depth, Fail Secure, \\
        Secure-by-Design\dots \\
        (concrete guidelines and methods)

    \subsection{Integrity \& Authenticity}

        e.g. Secure Boot, Firmware-Signatures, \\
        Update-Authenticity, Rollback Protection\dots

    \subsection{Isolation \& Access Control}

        e.g. Memory Protection, Task-Isolation, \\
        Scheduling-Isolation, Kernel Architecture\dots

    \subsection{Lifecycle Security Requirements}

        e.g. Provisioning, Update-Mechanisms, \dots \\
        (requirements and not specific implementations)

\section{Overview of IoT Operating Systems}

    "How do selected IoT OSes implement these principles and requirements? \\
    What are their differences, strengths, and weaknesses?" \\

    OS Comparison Table could also help here.
    \cite{chandraOperatingSystemsInternet2016}

    \subsection{Known Vulnerabilities and Historical Attacks}

        e.g. related to Privilege Escalation, Memory Protection, \\
        Buffer overflow\dots

    \subsection{OS-specific Security Mechanisms}

        \subsubsection{RIOT Security Features}

            e.g. Secure-Boot, \\
            Cryptographic Libraries, Modularity,\dots

        \subsubsection{FreeRTOS Security Features}

            e.g. Memory Protection, Task Isolation, Secure Sockets,\dots

        \subsubsection{Zephyr Security Features}

            e.g. Kernel Mode Separation, MPU Abstractions, MCUboot,\dots

        \subsubsection{TinyOS Security Features}

            e.g. Component-based Architecture, Access Control,\dots

\section{Challenges \& Research Directions}

    "Why will it stay challenging to secure IoT OSes, \\
    and what are current research trends?" \\

    \subsection{Resource Constraints vs. Security Mechanisms}

        e.g. Isolation, Cryptography, Scheduling\dots \\

        Hardware \& Cost may not be a bottleneck of IoT Security.
        \cite{pearsonMisconceptionHardwareCost2019}

    \subsection{Lifecycle Management \& Update Strategies}

        e.g. Secure OTA, Rollback Protection, Maintenance\dots \\
        (and why these will remain challenging)

    \subsection{Formal Verification, Microkernels \& Trusted Execution Environments}

        Trusted Execution Environment: TrustZone.
        \cite{ngabonzizaTrustZoneExplainedArchitectural2016a}

        List of Trusted Execution Environments.
        How to still get around TEE.
        \cite{bognarIntellectualPropertyExposure}

        Capabilities, Applications of TEEs (also TrustZone).
        Exploits in TrustZone
        \cite{munozSurveyInsecurityTrusted2023}

    \subsection{Recent Trends in Research}

        e.g. Edge Security, rBPF on MCU, Trusted Execution Environments, \\
        Microkernel Isolation, Lightweight Cryptography\dots \\

        Features for code protection of isolation.
        \cite{bognarIntellectualPropertyExposure}

        Countermeasures for software-based, architecture-based,
        memory protection.
        \cite{munozSurveyInsecurityTrusted2023}

        Solutions to various security layers like
        hardware, system/firmware, network, data.
        \cite{pearsonMisconceptionHardwareCost2019}

        Research Directions of RTOSs for IoT.
        \cite{abdelsameaRealTimeOperating2016a}

\section{Conclusion \& Outlook}

    \subsection{Summary of Insights}

    \subsection{Future Developments in IoT OS Security}

        Possible Solutions for Architectural and ISA Issues,
        Privilege Separation, Memory Corruption,
        Code Disclosure, Memory-Safe Programming,
        Firmware Updates \& Future Directions
        \cite{tanSoKWheresComprehensive}

        More Examples for future Work.
        \cite{al-boghdadyPresenceTrendsCauses2021}

        Available Solutions to mitigate IoT threats.
        (Table 6)
        \cite{hossainHolisticAnalysisInternet2024}

    \subsection{Open Questions \& Ongoing Challenges}

        Open Challenges for TEEs.
        \cite{munozSurveyInsecurityTrusted2023}

        Open Challenges for IoT Devices.
        \cite{azizalkabirSecuringIoTDevices2023}

        Open Challenges in general IoT Security.
        \cite{hossainHolisticAnalysisInternet2024}

%
% LITERATURE
%

\nocite{*}
\bibliographystyle{IEEEtran}
\bibliography{tex-ressources/WAI-BIB}

\vspace{12pt}

\end{document}
